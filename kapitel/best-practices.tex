\section{Best-Practices}
\label{sec:best-practices}

\begin{wichtigbox}
Beschreibt und beantwortet die in Related Work erarbeitete Fragestellung indem
Best Practices vorgestellt/herausgearbeitet werden. Weiters wird unser
Vorgehen beschrieben. Und zwar wie iterativ eine immer kleinere Kadenz
erreicht werden kann. In den Unterkapiteln werden die einzelnen
Iterationsschritte vorgestellt.
\end{wichtigbox}

\subsection{Jahr auf Quartal}
\label{subsec:jahr-auf-quartal}

In Unternehmen die nur einmal im Jahr eine neue Version ihres Produktes
bereitstellen, kommen häufig lineare, nicht iterative Vorgehensmodelle zum
Einsatz. Eines der bekanntesten Beispiele dafür ist das Wasserfallmodell, dass
aus verschiedenen Phasen wie z.B.: der Analyse, dem Entwurf, der Realisierung
(Implementierung) und dem Testen besteht. Wenn man diese Art von langwierigen
Prozess verfolgt, dann ist es für ein Unternehmen relativ schwer auf
Veränderungen am weltweiten Markt rasch reagieren zu können. Aktualisiert man
nur einmal im Jahr das Produkt läuft man Gefahr, dass die verwendete
Technologie höchstwahrscheinlich schon längst wieder veraltet ist. Daher muss
ein Unternehmen um am Markt konkurrenzfähig zu bleiben, auf kürzere Release-
Zyklen setzen.

Die dabei nächst kürzere Iterationslänge würde z.B.: drei Monate betragen.
Dabei könnte ein Unternehmen einmal im Quartal eine neue Version ihrer
Software bereitstellen. Die große Herausforderung ist die bisherigen Prozesse
so anzupassen, um dieselben Probleme in kürzerer Zeit zu lösen. Eine
Vorgehensweise die dabei offensichtlich nicht funktionieren würde, ist nichts
im Unternehmen zu verändern, jedoch alle bisherigen Praktiken nun in drei
Monaten durchzuführen. Da diese Art der Komprimierung nicht funktionieren
kann, müssen daher fundamentale Techniken adaptiert werden um bestimmte
Aufgaben zu gewissen Zeitpunkten in der kürzeren Iteration durchzuführen.

\subsubsection{Automatisierte Akzeptanztests}
\label{minisec:automatisierte-akzeptanztests}
Manuelles Testen der Software ist zeitaufwendig und fehleranfällig. Hat man
eine Iterationslänge von einem Jahr, kann man jedoch ohne weiteres diese Art
von Testen durchführen. Nach Beendigung der Implementierungsphase wird die
Software an die QA-Abteilung weitergeleitet, die dann mit dem ausführlichen
verifizieren des Produkts beschäftigt ist. Hat ein Unternehmen jedoch nur drei
Monate Zeit, dann wäre es zu aufwändig diesen Prozess jedes Quartal
wiederholen zu müssen. Die Automatisierung einer Regressions-Test-Suite, die
auf \emph{Knopfdruck} ausgeführt werden kann, hilft dem Unternehmen dabei
manuelle Tests aus dem Entwicklungsprozess zu entfernen. Dabei erhalten die
Entwickler nach kürzester Zeit Feedback über das Funktionsverhalten des
Systems. Diese Tests können in Form von Akzeptanztests realisiert werden. Bei
einer Iterationslänge von drei Monaten muss die endgültige Form bzw.
Geschwindigkeit der Tests nicht perfekt oder äußerst schnell sein. Wesentlich
ist nur, dass diese Tests automatisiert durchgeführt werden können. Die
entstehenden Wartezeiten sind dabei für den Projekterfolg nicht kritisch.

\subsubsection{Refactoring}
\label{minisec:refactoring}
Im Wasserfallmodell gibt es eine eigene Phase für den Entwurf der Software.
Möchte man alle drei Monate eine neue Version der Software bereitstellen, muss
das Designen der Software-Architektur über die gesamte Iteration verteilt
werden. Die Zeit für eine eigene Entwurfs-Phase steht bei dieser
Iterationslänge nicht zur Verfügung. Da es jetzt kein \emph{Big Design Up-
Front} geben kann, und das Design kontinuierlich den Gegebenheiten angepasst
werden muss, müssen Softwareentwickler die Technik des \emph{Refactorings}
ausgezeichnet beherrschen. Dadurch wird es möglich, große Design-Änderungen in
kleinen, sicheren Schritten durchzuführen ohne jedoch dabei das
Systemverhalten zu verändern. Entwickler übernehmen dabei die Verantwortung,
regelmäßig in den Softwareentwurf des Produkts zu investieren.

\subsubsection{Continuous Integration}
\label{minisec:continuous-integration}
Auch bei dieser Technik gilt dieselbe Argumentation wie auch schon beim
Refactoring. In einer dreimonatigen Iteration bleibt keine Zeit für eine
eigene Integrationsphase der Software. Sollte es z.B.: kurz vor einem neuen
Release zu Problemen bezüglich des Zusammenbaus des Produkts kommen, die von
den Entwicklern während der Implementierung nicht berücksichtigt wurden,
könnte eventuell ein weiterer Release-Zyklus erforderlich sein, diese um all
diese Probleme zu beheben. Deshalb muss eine Möglichkeit für die
kontinuierliche Integration der Software in Form eines Build-Servers
geschaffen werden, auf dem die Entwickler täglich ihre erledigten Aufgaben
hinzufügen können. Dadurch kann es am Ende der Iteration zu keinen
überraschenden Komplikationen bezüglich des Gesamtprodukts kommen.

\subsubsection{Subscription Modell}
\label{minisec:subscription-modell}
Muss ein Unternehmen nur einmal im Jahr den potentiellen Endkunden von der
neuen Version der Software überzeugen, fällt dieses Vorgehen alle drei Monate
deutlich schwieriger aus. Man kann den Kunden als Unternehmen nicht dazu
bringen, jedes Quartal für eine Aktualisierung der Software erneut zahlen zu
lassen. Sollte das Geschäftsmodell der Firma jedoch vorsehen, dass der Kunde
für Upgrades des Produkts zahlen muss, dann kann die Iterationslänge nicht auf
häufigere Releases umgestellt werden. Leider verliert man dadurch auch die
Vorteile des häufigeren Feedbacks des Benutzers, genauer gesagt all die
Informationen die man aufgrund der Benutzung des Produkts durch den Endkunden
und die Entwicklung des Produkts am Markt erhält. Daher muss das
Geschäftsmodell des Unternehmens ebenfalls angepasst werden und möglicherweise
eine Form von Subscription-Modell eingeführt werden. Dabei zahlt der Kunde
einmal im Jahr einen Pauschalbetrag und erhält sämtliche Upgrades der Software
ohne weitere Bezahlung. Diese Art von Geschäftsmodell ist absolut kritisch für
den Erfolg für eine Umstellung auf dreimonatige Release-Zyklen.


\subsection{Quartal auf Monat}
\label{subsec:quartal-auf-monat}

Bei jedem Übergang von einem längeren zu einem kürzeren Release-Zyklus ist es
notwendig, gewisse bis jetzt vielleicht erfolgreich eingeführte Praktiken zu
entfernen und neue zu adaptieren.

\subsubsection{Programmierer schreiben Tests}
\label{minisec:programmierer-schreiben-tests}
Hat man bei einer dreimonatigen Iteration ca. 60 Arbeitstage Zeit die neue
Version der Software zu entwickeln, ist es zeitlich nicht weiter tragisch,
falls die Ausführungsgeschwindigkeit der Akzeptanztests einen Tag erfordern.
In einem monatlichen Zyklus muss jedoch die Häufigkeit des Feedbacks für den
Entwickler drastisch erhöht werden. Dabei muss ein Teil der
Verifikationsarbeit den Programmierern übergeben werden, sodass diese in
kürzeren Zeitabständen Informationen über den Zustand der Software einholen
können.

Ein monatlicher Release-Zyklus macht es erforderlich, dass Entwickler selbst
Tests schreiben und auch ausführen. Daraus kann man schließen, dass die Anzahl
der Akzeptanztests nicht nur reduziert werden kann, sondern auch nicht mehr
alle eventuellen Fehler aufdecken müssen, da diese bereits vorher von den
Unit-Tests abgefangen werden.

\subsubsection{Status Meetings}
\label{minisec:status-meetings}

Auch die Art und Weise wie man andere Teammitglieder über durchgeführte
Veränderungen informiert muss bei monatlichen Iterationen angepasst werden.
Wurden vielleicht bis jetzt über alle Aktualisierungen der Software Protokolle
für den Projekt-Manager geschrieben, der diese wiederum an andere Entwickler
als Feedback weiterleitete, muss man nun eine Form des Wissenstransports
schaffen, der bei weitem nicht so viel Zeit in Anspruch nimmt. Hat man nur
noch 20 Arbeitstage für die Entwicklung eines Upgrades der Software, dann hat
diese Art des relativ \emph{schwergewichtigen}, formalen Prozesses keine
Daseinsberechtigung.

Man benötigt daher eine Form der täglichen Status-Aktualisierung über
Projektveränderungen der einzelnen Teammitglieder. Das kann z.B.: in Form
eines täglichen Stand-Up oder Daily-Scrum Meetings erfolgen bei dem jeden
Morgen jeder im Team kurz über Neuigkeiten bzw. eventuelle Probleme berichtet.

\subsubsection{Task Board}
\label{minisec:task}
Klassische Planungsprozesse bei denen zu erledigende Aufgaben mehrere
Stationen (z.B.: Projektmanager, Analysten, etc.) durchwandern müssen, danach
noch eventuell in Form eines Berichts niedergeschrieben werden, bevor sie der
Programmierer zu Gesicht bekommt um daran arbeiten zu können, müssen ebenfalls
angepasst werden.

Um Sinnvoll innerhalb eines Monats Planen zu können, werden daher
transparente, visuelle Techniken benötigt die außerdem noch öffentlich
zugänglich sein müssen. Dabei kann eine Art von Stellwand (task board) in
Kombination von Karteikarten, wie sehr oft in Scrum Verwendung finden,
eingesetzt werden. Dabei repräsentieren die Karten die durchzuführenden
Aufgaben die jeweils in entsprechenden Zustandspalten auf der Stellwand
platziert werden. Somit hat jedes Teammitglied zu jedem Zeitpunkt der
Iteration eine Übersicht, welche Aufgaben noch innerhalb dieses Zyklus zu
erledigen sind.

\subsubsection{Pay-per-Use Modell}
\label{minisec:pay-per-use-modell}

Bei monatlichen Releases kann es erneut sinnvoll sein über eine Anpassung des
Geschäftsmodells nachzudenken.  Bei dieser Iterationskürze könnte das \emph{Pay-
per-Use}-Modell eingeführt werden. Bei dreimonatigen Prozessen kann diese Art
von Geschäftsmodell gefährlich sein. Sollte ein Release fehlerhaft sein,
könnte das die Einnahmen des Unternehmens verringern, jedoch könnten die
Entwickler erst drei Monate später darauf reagieren.

In einmonatigen Prozessen können viel schneller Korrekturen vorgenommen
werden. Außerdem kann die Information über die tatsächliche Benutzung des
Produkts als wertvolles Feedback angesehen werden. Bei Geld handelt es sich
jedoch mit Abstand um das beste Feedback das man außerdem wieder in das
Unternehmen investieren kann.

\subsubsection{Notwendige Entfernung von Praktiken}
\label{minisec:monat-entfernte-praktiken}
Auch bei dieser Geschwindigkeitsüberführung ist es offensichtlich nicht
möglich, in der gleichen Art und Weise Software zu entwickeln wie bisher.
Jedoch gibt es gewisse Übereinstimmungen wie z.B.: die Akzeptanztests, die
allerdings in einem einmonatigen Zyklus wesentlich schneller ablaufen müssen.
Es fällt auf, dass  gewisse Aufgaben die zuvor vielleicht nur von einer Person
durchgeführt wurden, in kürzeren Iterationen von mehreren Teammitgliedern
erledigt werden müssen. Auf die Häufigkeit der Durchführung und der
Durchführungszeitpunkt verändern sich.

Allerdings wurden in der Überführung von jährlichen zu dreimonatigen Zyklen
weitere notwendige Praktiken eingeführt, für die in einmonatigen Phasen keine
Zeit mehr vorhanden ist. Diese Techniken waren äußerst hilfreich um die
Entwicklungsgeschwindigkeit in einem ersten Verkürzungsprozess zu erhöhen. Sie
haben dabei geholfen regelmäße Upgrades der Software bereitzustellen, jedoch
war das Erlernen und Einhalten der Techniken für jedes Teammitglied sehr
aufwändig.

\subsubsection{QA-Abteilung}
\label{minisec:qa-abteilung}

Bei einmonatigen Iterationen werden diese Praktiken jedoch zur Last. Z.B.: ist
das Vorhandensein einer QA-Abteilung aufgrund der organisatorischen Entfernung
nicht mehr möglich. Diese Abteilung darf jedoch nicht mit der Rolle der Tester
verwechselt werden.

Natürlich darf auch der psychologische Effekt bezüglich der Reduzierung
dieser in längeren Iterationen noch so wichtigen Einrichtung nicht vergessen
werden.  Ein Entwickler der viel Erfahrung mit dreimonatigen Release-Zyklen
besitzt, für den die QA-Abteilung der erste Schritt aus dem Chaos war, d.h.
man tatsächlich Software am Ende des Quartals bereitstellen konnte, die noch
dazu für den Kunden problemlos funktioniert hat, für diesen Entwickler ist
eine QA-Abteilung unverzichtbar. Er kann sich nicht vorstellen, wie man in
einem Monat Software erfolgreich bereitstellen soll, ohne der Unterstützung
dieses \emph{Fehlerfangnetzes}. Aus der Perspektive dieses Entwicklers sind
dessen Argumentationen völlig nachvollziehbar.

Betrachtet man jedoch den folgenden Prozess genauer, bei dem jede
Funktionsänderung der Software über den Projektmanager zur QA-Abteilung
weitergeleitet wird, damit diese dann notwendige Ressourcen allokieren kann um
die erhaltene Anfrage bearbeiten zu können, erkennt man, dass man in einem
Monat nicht genug Zeit für diese Vorgehensweise hat.

Obwohl die QA-Abteilung die erhaltenen Anfragen aus Effizienzgründen in
Warteschlangen organisiert damit dann ein Tester diese Aufgabe entnehmen und
bearbeiten kann, dauert es für den Entwickler viel zu lange, bis er endlich
Feedback erhält um darauf reagieren zu können. Wenn man jetzt noch darüber
nachdenkt, dass die Programmierer die zurückbekommenen Antworten der Tester
ebenfalls ähnlich organisieren und zum Beheben derselbe Prozess erneut
durchgeführt werden muss kommt man zum Schluss, dass die Iteration bereits zu
Ende ist bevor überhaupt nur eine Aufgabe abgeschlossen wurde. Man könnte
natürlich die Software trotz der Tatsache bereitstellen, dass man nicht 100
Prozentig sicher weiß, ob das System Fehler enthält oder nicht. Allerdings ist
das Bereitstellen von fehlerhaften Upgrades für den Endkunden auf Dauer nicht
tragbar.

\begin{wichtigbox}
TODO: Fertig ausformulieren\\
D.h., dass das Q/A-Department, das kritisch für den Erfolg für jährliche bzw.
dreimonatige Deployments war, wird zur unüberwindbaren Barriere in
einmonatigen Prozessen und muss deshalb verworfen werden. Und wieder müssen
dieselben Probleme nur jetzt ohne Q/A-Department erledigt werden also was
machen? Ganz einfach, Tester müssen dem Entwicklungsteam hinzugefügt werden.
D.h. im Endeffekt hat man einen großen Raum in denen sich sowohl Tester als
auch Entwickler befinden und diese unmittelbar miteinander kommunizieren
können.

Nichts von den Vorgehensweisen über Warteschlangen, nichts von dem
zeitaufwendigen weiterleiten von Requests über verschiedene Personen, sondern
die unmittelbare Kommunikation zwischen Tester und Entwickler sind notwendig.
Z.B. nimmt ein Entwickler Veränderungen an der Benutzerschnittstelle vor und
berichtet dem Tester davon. Dieser kann sofort überprüfen ob die vorgenommen
Änderungen fehlerfrei funktionieren. D.h. der Entwickler erhält nur wenige
Augenblicke später sofortiges Feedback des Testers.
\end{wichtigbox}

\subsection{Monat auf Woche}
\label{subsec:monat-auf-woche}

\subsection{Woche auf Tag}
\label{subsec:subscription-modell}
