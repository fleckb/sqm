\subsection{Woche auf Tag}
\label{subsec:subscription-modell}

Neben den technischen Herausforderungen für tägliche Releases erhält der
Entwickler den angenehmen Nebeneffekt, dass die tägliche Arbeit bereits am
nächsten Tag von irgendeinem Kunden bereits verwendet wird. Diese Tatsache
darf als Motivationsfaktor für den Softwareentwickler nicht unterschätzt
werden. Sehr oft haben Entwickler das Gefühl, dass ihre erledigte Arbeit für
niemanden spürbar von Bedeutung sind.

\subsubsection{Immunization}
\label{minisec:immunization}
Bei täglichen Deployments ist diese Arbeit bereits am nächsten Tag von
Bedeutung. Natürlich stellt sich dabei die Frage wie mit Fehlern umgegangen
wird. Ist es überhaupt noch möglich bei dieser Kürze der Iteration Fehler
Softwareentwickler zu akzeptieren? Sichtlich kann es sich ein Entwickler nicht
mehr leisten, einen schlechten Tag zu haben.

D.h. sollte ein Team-Mitglied z.B. die Nacht zu vor zu wenig Schlaf bekommen
haben und daher das Gefühl haben es sollte heute nicht programmieren, dann
sollte es sich seinem Gefühl fügen. Dabei handelt es sich tatsächlich im
Endeffekt um einen Produktivitätsgewinn wenn ein Entwickler bei schlechter
Tagesverfassung nicht echten Code implementiert.

Sollte doch entwickelt werden und es werden Fehler erzeugt, dann sieht man
diese unmittelbar am nächsten Tag, nämlich nachdem der Kunde die fehlerhafte
Version der Software verwendet hat. Die Kernaussage ist, wenn man sich nicht
im Stande fühlt 100 Prozentig konzentriert zu Programmieren, dann sollte man
es bei täglichen Releases auch nicht tun.

Es gibt sicherlich genug andere Aufgaben die an solchen Tagen verrichtet
werden kann und sei es nur jemanden anderen bei seiner Aufgabe zu
unterstützen. Man kann sich metaphorisch dabei vorstellen, dass man bei
täglichen Deployments das brüchige Eis unter seinen Zehenspitzen fühlt. Man
darf nicht zu stark auftreten sonst bricht diese dünne Eisschicht sofort unter
einem zusammen. Um all dies jedoch umsetzen zu können, muss es unmittelbar
nach einem fehlerhaften Deployment möglich sein, automatisch eine frühere
Version der Software herzustellen. Diese Art des Rollbacks nennt man
Immunization.

Eine weitere Variante wie man mit fehlerhaften Releases umgehen kann ist, dass
nach einem Deployment eine Art Monitor der Software automatisch das Team
benachrichtigt, dass es ein Problem mit dem System gibt. Daraufhin stoppen
alle Entwickler ihre aktuellen Arbeiten und beheben den Fehler worauf eine
neue Version der Software deployt werden kann.

\subsubsection{Data-Informed-Usability}
Man trifft bei täglichen Releasese dieselben Usability-Entscheidungen aber
diesmal kann man mit Hilfe von echten Daten entscheiden. A/B-Testing ist eine
Art von diesem Vorgehen bei dem z.B. an einem Tag eine bestimmte Version der
Software released wird und danach Daten eingeholt werden. Am nächsten Tag
deployt man eine abgeänderte Version und analysiert wieder die entstandenen
Informationen. Z.B. könnte man die Transaktionenanzahl messen oder wie lange
Benutzer auf einer Seite geblieben sind. Man kann also die Daten die man
generiert dazu verwenden, um Usability-Entscheidungen zu treffen. Somit lassen
sich UI-Konflikte  aufgrund von messbaren Daten vermeiden.

\subsubsection{Feature-Flags}
\label{minisec:featureflags}
Die Idee dabei ist, dass man zur Laufzeit wenn man z.B. die Software auf 100
Server verteilt hat, und nebenbei hat man eine Konsole, die es jemanden
erlaubt Features ein- und auszuschalten. D.h. man eine globale Anzahl von
Flags, die es möglichen machen, gewisse Funktionalitäten für gewisse Benutzer
freizuschalten bzw. zu deaktivieren.

Dabei kann man taktisch vorgehen wenn man z.B. ein neues Features
implementiert jedoch zehn weitere neue Features ebenfalls mit deployt werden.
Nun deaktiviert man sein Features mittels Konsole, wartet eine Zeit lang ab
bis die anderen Features ihre Wirkung gezeigt haben und aktiviert das sein
Feature wieder. Jetzt kann man den Output analysieren den das Feature
verursacht hat. Man erkennt leicht, dass bei täglichen Deployments die
Laufzeitkonfiguration der Software wirklichen Mehrwert einbringt.

\begin{lstlisting}[float=h!tb,caption=Beispiel eines Feature Flags bei Flickr \cite{flickr09}, label=fflagcode]
if ($cfg.enable_unicorn_polo) {
    // do something new and amazing here.
}
else {
    // do the current boring stuff.
}
\end{lstlisting}

\subsubsection{One-Piece-Flow}
Kanban ist derzeit in der Entwickler-Community sehr beliebt. Einer der Gründe
ist, dass man den Arbeits-Flow des gesamten Teams auf einen Blick erkennen
kann. Dabei limitiert man die Arbeit die in Bearbeitung ist und kennzeichnet
dies anhand von Karteikarten auf einem sogenannten Kanban-Board. Dabei besteht
dieses Board aus mehreren Spalten die eine gewisse Bedeutung haben wie z.B.
die Aufgaben die gerade in Arbeit sind oder die, die gerade getestet werden.

Bei täglichen Deployments hat man allerdings nicht mehr die Zeit, Aufgaben von
einer Spalte zur nächsten weiterzureichen. Entwickler treffen eine hohe Anzahl
an verschiedenen Entscheidungen bei täglichen Deployments im Gegensatz zu
längeren Release-Zyklen. Das liegt daran, dass die Entwickler wirklich
schnelles Feedback benötigen um entsprechend schnell darauf reagieren zu
können. Dabei nehmen sie viele verschiedene Rollen ein z.B. die des Designers
oder des Implementierers oder die des Performance-Testers, etc. D.h. die
einzelnen Stationen bei Kanban werden zu einem \emph{One-Piece-Flow} wobei der
Entwickler dafür die Verantwortung übernehmen muss, neue Funktionalitäten
durch all diese Stationen zu führen um sie anschließend zu releasen.

\subsubsection{Multi-Level-Staging}
Bei täglichen Releases bleibt keine Zeit mehr, die Software durch verschiedene
Stages zu führen. D.h. der Flow von einer Entwickler-Box über eine
Integrations-Box wo einige Tests durchgeführt werden, über eine Testing-Box,
über eine Pre-Production-Box und zu guter Letzt die Produktionsumgebung.
Software durch all diese Stationen zu führen, benötigt zu viel Zeit wenn man
täglichen deployn möchte.

D.h. man muss wie immer dieselben Problemen in kürzerer dadurch lösen, indem
man Verantwortungen verändert damit man dasselbe Feedback bekommt nur basiert
auf einer kürzeren Pipeline. D.h. ideal wären eine Development-Box und eine
Produktionsumgebung.

\subsubsection{Operations Department}
Heutzutage gibt es noch sehr oft eine Trennung zwischen dem Operations-
Department und dem Entwicklungs-Team in einem IT-Unternehmen. Dabei mussten
sich Entwickler nicht um Infrastrukturangelegenheiten kümmern. Bei täglichen
Releases lässt sich diese Aufgabe jedoch nicht mehr so einfach delegieren.
Wieder einmal gibt es nicht genug Zeit für den Roundtrip zwischen Operations-
Team und Entwickler-Team. Natürlich gibt es auch weiterhin Operations-
Engineers deren Job es ist, das System zu monitoren und gewisse Prozesse zu
automatisieren und Entwickler zu schulen um Frameworks und Tools zu erstellen,
sodass jeder Verantwortung für Operations-Angelegenheiten übernehmen kann.

\subsubsection{Keine Status Meetings}
Obwohl es eine der wichtigen agilen Praktiken ist, bleibt einem Entwickler
keine Zeit darauf zu warten, dass man von jemanden anderen am nächsten Tag im
Stand-Up Meeting erfährt, wie weit er mit seiner Aufgabe ist. Waren sie bei
monatlichen Releases eine tolle Sache, so sind sie bei täglichen Iterationen
eher ein Hindernis.

Natürlich muss weiterhin kommuniziert werden, jedoch auf eine andere Art und
Weise. Dabei müssen die Entwickler natürlich im selben Raum sitzen und mit der
Tatsache klar kommen, dass man bei der Arbeit unterbrochen wird. Vielleicht
hat man auch einen IRC-Channel, irgendeine Art von Social-Media wie eine
Facebook-Group die einem eine Real-Time Übersicht über die Arbeit eines jeden
einzelnen Entwicklers gibt zu jedem beliebigen Zeitpunkt gibt.
