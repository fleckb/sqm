\subsection{Woche auf Tag}
\label{subsec:subscription-modell}

Neben den technischen Herausforderungen für tägliche Releases erhält der Entwickler den angenehmen Nebeneffekt, dass die tägliche Arbeit bereits am nächsten Tag von irgendeinem Kunden bereits verwendet wird. Diese Tatsache darf als Motivationsfaktor für den Softwareentwickler nicht unterschätzt werden. Sehr oft haben Entwickler das Gefühl, dass ihre erledigte Arbeit für niemanden \endquote{spürbar} von Bedeutung ist.

\subsubsection{Immunization}
\label{minisec:immunization}
Bei täglichen Deployments ist diese bereitgestellte Funktionalität schon am nächsten Tag für den Kunden anwendbar. Natürlich stellt sich dabei die Frage wie dabei mit Fehlern umgegangen wird. Ist es überhaupt noch möglich bei dieser Kürze der Iteration Fehler von Softwareentwicklern zu akzeptieren? Hat Teammitglied einmal einen schlechten Tag, das heißt, es fühlt sich nicht besonders, können dabei Fehler produziert werden, die zu fatalen Folgen führen. 

Wenn ein Entwickler das Gefühl haben es sollte an diesem Tag lieber nicht zu programmieren, dann sollte er es auch wirklich nicht tun. Dabei handelt es sich im Endeffekt um einen Produktivitätsgewinn wenn bei schlechter Tagesverfassung kein Produktionscode implementiert wird. 

Sollte doch trotz schlechter Tagesverfassung entwickelt werden und es werden Fehler erzeugt, dann sieht der Kunde diese unmittelbar am nächsten Tag, nämlich nachdem er die fehlerhafte Version der Software verwendet hat. Die Kernaussage ist, wenn man sich nicht im Stande fühlt 100 Prozentig konzentriert zu programmieren, dann sollte man es bei täglichen Releases auch nicht tun. 

Es gibt sicherlich genug andere Aufgaben die an solchen Tagen verrichtet werden kann und sei es nur jemanden anderen bei seiner Aufgabe zu unterstützen. Man kann sich metaphorisch dabei vorstellen, dass man bei täglichen Deployments das brüchige Eis unter seinen Zehenspitzen fühlt. Man darf nicht zu stark auftreten sonst bricht diese dünne Eisschicht sofort unter einem zusammen. 

Um all dies jedoch umsetzen zu können, muss es unmittelbar nach einem fehlerhaften Deployment möglich sein, automatisch eine frühere Version der Software herzustellen. Diese Art des Rollbacks nennt man Immunization.

Eine weitere Variante wie man mit fehlerhaften Releases umgehen kann ist, dass nach einem Deployment eine Art Monitor der Software automatisch das Team benachrichtigt, dass es ein Problem mit dem System gibt. Daraufhin stoppen alle Entwickler ihre aktuellen Arbeiten und beheben den Fehler worauf eine neue Version der Software deployt werden kann.

\subsubsection{Data-Informed-Usability}
Man trifft bei täglichen Releasese dieselben Usability-Entscheidungen aber diesmal kann man mit Hilfe von echten Daten entscheiden. A/B Testing ist eine Art von diesem Vorgehen bei dem zum Beispiel an einem Tag eine bestimmte Version der Software released wird und danach Daten eingeholt werden \cite{webanalytics2009}. Am nächsten Tag deployt man eine abgeänderte Version und analysiert wieder die entstandenen Informationen. Zum Beispiel könnte man die Transaktionsanzahl messen oder wie lange Benutzer auf einer Seite geblieben sind. 

Man kann also die Daten die man generiert dazu verwenden, um Usability-Entscheidungen zu treffen. Somit lassen sich UI-Diskussionen die auf rein subjektiven Tatsachen beruhen und eventuell zu Konflikten führen können, aufgrund von messbaren Daten vermeiden.

\subsubsection{Feature-Flags}
\label{minisec:featureflags}
Die Idee dabei ist, dass man eine globale Anzahl von Flags konfiguriert hat, die es möglichen machen, gewisse Funktionalitäten für gewisse Benutzer freizuschalten bzw. zu deaktivieren \cite{weboperations2010}. Zum Beispiel kann ein Entwickler über eine Konsole neue Features für eine bestimmte Anzahl an Servern, auf denen die Software installiert ist, ein- beziehungsweise ausschalten. Dabei sind die restlichen Server von dieser Änderung nicht betroffen und funktionieren wie bisher.

Dabei kann natürlich taktisch vorgegangen werden. Möchte ein Entwickler die Auswirkungen seines implementierten Features auf die Software messen, jedoch weitere Features beim nächsten Release mitdeployt werden, kann das eigene Feature vorerst deaktiviert werden. Nachdem dann das Update mit all seinen Änderungen Wirkung gezeigt hat, kann der Entwickler das Feature aktivieren und damit beginnen die Daten zu analysieren die das Feature verursacht hat. 

Man erkennt leicht, dass bei täglichen Aktualisierungen der Software die Laufzeitkonfiguration der Software wirklichen Mehrwert einbringt.

\begin{lstlisting}[float=h!tb,caption=Beispiel eines Feature Flags bei Flickr \cite{flickr09}, label=fflagcode]
if ($cfg.enable_unicorn_polo) {
    // do something new and amazing here.
}
else {
    // do the current boring stuff.
}
\end{lstlisting}

\subsubsection{One-Piece-Flow}
Kanban ist derzeit in der Entwickler-Community sehr beliebt. Einer der Gründe ist, dass man den Arbeits-Flow des gesamten Teams auf einen Blick erkennen kann. Dabei limitiert man die Arbeit die in Bearbeitung (Work in Progress) ist und kennzeichnet dies anhand von Karteikarten auf einem sogenannten Kanban-Board. Dabei besteht dieses Board aus mehreren Spalten die eine gewisse Bedeutung haben, wie zum Beispiel die Aufgaben die gerade in analysiert werden oder die, die gerade getestet werden. 

Bei täglichen Deployments hat man allerdings nicht genügend Zeit, Aufgaben von einer Spalte zur nächsten weiterzureichen. Entwickler treffen bei täglichen Deployments eine hohe Anzahl an verschiedenen Entscheidungen, im Gegensatz zu längeren Iterationen. Das liegt daran, dass die Entwickler unmittelbares Feedback benötigen um entsprechend schnell darauf reagieren zu können.

Dabei nehmen sie viele verschiedene Rollen ein. Zum Beispiel die Rolle des Designers oder des Programmierers oder die des Performance-Testers, etc. Das heißt, die einzelnen Stationen bei Kanban werden zu einem \endquote{One-Piece Flow} wobei der Entwickler selbst dafür die Verantwortung übernehmen muss, neue Funktionalitäten durch all diese Stationen zu führen um sie anschließend freizugeben \cite{sutherland2011}.

\subsubsection{Multi-Level-Staging}
Bei täglichen Releases bleibt keine Zeit mehr, die Software durch verschiedene Stages zu führen. Das heißt, der Flow von einer Entwickler-Box über eine Integrations-Box, wo einige Tests durchgeführt werden, über eine Testing-Box, über eine Pre-Production-Box und zu guter Letzt die Produktionsumgebung, ist nicht mehr möglich. Software durch all diese Stationen zu führen benötigt zu viel Zeit wenn man täglich neue Versionen bereitstellen möchte. 

Das heißt, man muss wie immer dieselben Problemen in kürzerer dadurch lösen, indem man Verantwortungen verändert damit man dasselbe Feedback bekommt nur basiert auf einer kürzeren Pipeline. Ideal wären die Verwendung einer Development-Box und einer Produktionsumgebung.

\subsubsection{Operations Department}
Heutzutage gibt es noch sehr oft eine Trennung zwischen dem Operations-Department oder auch System-Administration genannt, und dem Entwicklungsteam in einem IT-Unternehmen. Dabei mussten sich Entwickler nicht um Infrastrukturangelegenheiten kümmern. Bei täglichen Zyklen lässt sich diese Aufgabe jedoch nicht mehr so einfach delegieren. 

Bei täglichen Releases reicht die Zeit nicht mehr um den Umweg zwischen Operations-Team und Entwicklerteam zu unterstützen. Natürlich gibt es auch weiterhin Infrastruktur-Spezialisten deren Job es ist, das System zu überwachen, gewisse Prozesse zu automatisieren, Entwickler zu schulen und um Frameworks und Tools zu erstellen, sodass jeder Verantwortung für Operations-Angelegenheiten übernehmen kann \cite{weboperations2010}.

\subsubsection{Keine Status Meetings}
Obwohl es eine der wichtigsten agilen Praktiken ist, bleibt einem Entwickler bei täglichen Releases keine Zeit einen ganzen Tag darauf zu warten, dass man über den aktuellen Arbeitsfortschritt eines jeden Teammitglieds informiert wird. Waren Stand-Up Meetings bei monatlichen Iterationen absolute Voraussetzung um Status-Updates zu bekommen, so sind sie bei täglichen Zyklen eher ein Hindernis. 

Natürlich muss weiterhin intensiv innerhalb des Teams kommuniziert werden, jedoch auf eine andere Art und Weise. Grundvoraussetzung dabei ist, dass die Entwickler im selben Raum sitzen und mit der Tatsache klar kommen, dass man ständig bei der Arbeit unterbrochen werden kann. Vielleicht hat man auch einen IRC-Channel, irgendeine Art von Social-Media wie eine Facebook-Group die einem eine Real-Time Übersicht über die Arbeit eines jeden einzelnen Entwicklers zu jedem beliebigen Zeitpunkt gibt. 
