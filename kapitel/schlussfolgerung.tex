\section{Schlussfolgerungen}
\label{sec:schlussfolgerungen}

\begin{sidewaystable}[p]
\sffamily
\caption{Übersicht über die Best Practices}
\label{tab:uebersicht}
\begin{tabular}{p{2.5cm}*{4}{p{4cm}}}
\toprule
& \textbf{Jahr auf Quartal} & \textbf{Quartal auf Monat} & \textbf{Monat auf Woche} & \textbf{Woche auf Tag} \\
\midrule
\multirow{5}{5em}{\textbf{Or\-ga\-ni\-satorische Aspekte}}
& & Programmierer schreiben Tests & Temporäre Branches       & One-Piece-Flow \\
& & Status Meetings               & Kanban                   & kein Operations Department \\
& & Task Board                    & Kein separates Test Team & keine Status Meetings\\
& & keine QA-Abteilung            & Keine Up-front Usability & \\
& & kein Change Management        & & \\
\addlinespace
\multirow{3}{5em}{\textbf{Werkzeug- und Methoden\-unterstützung}}
& Automatisierte Akzeptanztests & keine mehrfach releasten Versionen       & Keystoning                  & Immunization \\
& Refactoring                   & keine separaten Design Dokumente         & One-Button-Deploy           & Data-Informed-Usability \\
& Continuous Integration        & keine separaten Build- und Analyse-Teams & Kein aktiver Release-Branch & Feature-Flags \\ 
&                               &                                          &                             & kein Multi-Level-Staging \\
\addlinespace
\textbf{Geschäftliche Aspekte} & Subscription Modell & Pay-per-Use Modell & & \\
\bottomrule
\end{tabular}
\end{sidewaystable}

Siehe Tabelle~\ref{tab:uebersicht} auf Seite~\pageref{tab:uebersicht}.

\begin{wichtigbox}
Kurze Zusammenfassung der \enquote{Ergebnisse}, bzw. Auflisten der Vorteile
wenn man sich an die genannten Best Practices hällt. Dabei kann man sich auch
stark an dem Abstract orientieren.
\end{wichtigbox}

\begin{quote}
\zitat{Does this mean we're never going to introduce bugs to our live site? Of
course not - but we're going to keep the number of bugs to hit the live site
to a minimum, and we've made it easy and fast to get bug fixes live as
well.}~\cite{digg4}
\end{quote}
