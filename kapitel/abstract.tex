\section*{Abstract}

Diese Arbeit besch�ftigt sich mit der Ausarbeitung von qualit�tssichernden Ma�nahmen, die es Software-Unternehmen erm�glichen sollen, Deploymentzyklen drastisch zu verk�rzen um rasch auf Ver�nderungen am Markt reagieren zu k�nnen. IT-Unternehmen die nur einmal j�hrlich eine neue Version ihrer Produkte zur Verf�gung stellen, k�nnten k�nftig gro�e Probleme haben konkurrenzf�hig zu bleiben. Diese Arbeit befasst sich daher mit der Frage, welche Entwicklungspraktiken zu einem bestimmten Vorgehensmodell hinzugef�gt bzw. von diesem entfernt werden m�ssen um die Deploymentgeschwindigkeit von j�hrlichen auf dreimonatige, auf monatliche, auf w�chentliche, auf t�gliche und zu guter Letzt auf st�ndliche Zyklen zu verk�rzen. Da sich die Disziplin des Software-Engineerings nicht verallgemeinern l�sst, k�nnen bestimmte Techniken f�r bestimmte Deploymentzyklen positive, f�r andere jedoch negative Auswirkungen haben. Wenn man sich vorstellen w�rde, dass man zwei Entwicklungsteams, die mit unterschiedlich langen Deploymentzyklen arbeiten, nach ihren Praktiken befragen w�rde, was w�rde man als Antwort bekommen? Eines ist klar, Software-Entwickler m�ssen prinzipiell dieselben Probleme l�sen und zwar von der Idee bis zum tats�chlichen Bereitstellen des Produkts f�r den Endanwender. Nur die Art und Weise wie die Software umgesetzt wird unterscheidet sich dramatisch je nach L�nge des verwendeten Deploymentzyklus. In dieser Arbeit wollen wir detailliert darauf eingehen, welche Techniken notwendig sind, um einerseits hohe Qualit�t der Software zu garantieren und um andererseits die Entwicklungszyklen drastisch zu verk�rzen.
