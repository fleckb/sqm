\section*{Abstract}

Diese Arbeit beschäftigt sich mit der Ausarbeitung von qualitätssichernden
Maßnahmen, die es Software-Unternehmen ermöglichen sollen, Deploymentzyklen
drastisch zu verkürzen um rasch auf Veränderungen am Markt reagieren zu
können. IT-Unternehmen die nur einmal jährlich eine neue Version ihrer
Produkte zur Verfügung stellen, könnten künftig große Probleme haben
konkurrenzfähig zu bleiben. Diese Arbeit befasst sich daher mit der Frage,
welche Entwicklungspraktiken zu einem bestimmten Vorgehensmodell hinzugefügt
bzw. von diesem entfernt werden müssen um die Deploymentgeschwindigkeit von
jährlichen auf dreimonatige, auf monatliche, auf wöchentliche, auf tägliche
und zu guter Letzt auf stündliche Zyklen zu verkürzen. Da sich die Disziplin
des Software-Engineerings nicht verallgemeinern lässt, können bestimmte
Techniken für bestimmte Deploymentzyklen positive, für andere jedoch negative
Auswirkungen haben. Wenn man sich vorstellen würde, dass man zwei
Entwicklungsteams, die mit unterschiedlich langen Deploymentzyklen arbeiten,
nach ihren Praktiken befragen würde, was würde man als Antwort bekommen? Eines
ist klar, Software-Entwickler müssen prinzipiell dieselben Probleme lösen und
zwar von der Idee bis zum tatsächlichen Bereitstellen des Produkts für den
Endanwender. Nur die Art und Weise wie die Software umgesetzt wird
unterscheidet sich dramatisch je nach Länge des verwendeten Deploymentzyklus.
In dieser Arbeit wollen wir detailliert darauf eingehen, welche Techniken
notwendig sind, um einerseits hohe Qualität der Software zu garantieren und um
andererseits die Entwicklungszyklen drastisch zu verkürzen.
