\section{Success Stories / Case Studies}

\begin{wichtigbox}
Um dem Ganzen einen fundierten Unterbau zu geben ein paar Success
Stories vorstellen, welche bereits CD erfolgreich einsetzen. Gut w�re es hier
nat�rlich wenn es nicht nur Beispiele aus der Industrie sind, sondern
wissenschaftliche Ver�ffentlichungen.
\end{wichtigbox}

\subsection{IMVU}
%http://timothyfitz.wordpress.com/2009/02/10/continuous-deployment-at-imvu-doing-the-impossible-fifty-times-a-day/
%http://radar.oreilly.com/2009/03/continuous-deployment-5-eas.html
%http://www.infoq.com/news/2009/03/Continuous-Deployment
%http://www.developsense.com/blog/2009/03/50-deployments-day-and-perpetual-beta/#268747558129898245
%http://www.slideshare.net/bgdurrett/sds-2010-continuous-deployment-at-imvu

Als erstes Beispiel dient IMVU~\footnote{IMVU: \url{http://www.imvu.com/}},
eine soziale Online Community, in der in einer virtuellen Realit�t mit Hilfe
von 3D Avataren geplaudert, Spiele gespielt und eigene Inhalte erschaffen und
ausgetauscht werden k�nnen.

IMVU war eines der ersten \emph{Lean Startup} Unternehmen welche Continuous
Deployment aktiv einsetzten. Dabei ist dies aber nicht vorab im Ganzen geplant
worden, sondern inkrementell entstanden. Derzeit sind bei IMVU zirka 50
technische Mitarbeiter angestellt. Ein wichtiger Punkt warum bei vorallem so
vielen Entwicklern Continuous Deployment funktioniert, ist, dass es ein
zentraler Bestandteil der Firmenkultur ist. \\
Als Vorteile von Continuous Deployment werden von IMVU die folgenden Punkte
genannt:

\begin{itemize*}
    \item Regression wird sehr rasch erkannt
    \item Fehler k�nnen schneller behoben werden, da zwischen dem einspielen
          eines Fehlers und der Meldung �ber ein Problem nicht viel Zeit vergeht
    \item Der Release einer neuen Version erzeugt keinen zus�tzlichen Overhead
    \item Als Feedback bekommen sie sofort messbare Kerndaten von echten Kunden
\end{itemize*}

\minisec{Workflow} Eine wichtige Grundvoraussetzung f�r Continuous Deployment
bei IMVU ist wie schon in Abschnitt~\ref{subsec:jahr-auf-quartal} auf
Seite~\pageref{subsec:jahr-auf-quartal} gezeigt: Continuous Integration. Als
Technologie kommt hier Buildbot\footnote{Buildbot:
\url{http://trac.buildbot.net/}} zum Einsatz. Um die Vorteile von Continuous
Integration voll ausn�tzen zu k�nnen wird beim Entwickeln selbst \emph{Commit
Early Commit Often} praktiziert. Ist ein Feature fertig entwickelt, oder eine
Bug behoben worden, werden zuerst lokale  Tests auf der Entwicklermaschine
durchgef�hrt. Wenn all diese Tests positiv durchlaufen wurden, wird der neue
Code in Quellcode Versionsverwaltung eingespielt.

Erst jetzt werden s�mtliche Tests der Test-Suite angesto�en. Zur Zeit sind
dies zirka $15.000$ Tests aus den Bereichen Unit-Tests, Funktions-Tests und
Verhaltens-Tests. Dabei werden die folgenden Technologien eingesetzt:

\begin{itemize*}
    \item Selenium Core wird mit einem eigens entwickelten API Wrapper f�r
          die Verhaltens-Tests eingesetzt
    \item YUI Test wird f�r Browser bsierte JavaScript Unit-Tests verwendet
    \item PHP SimpleTest
    \item Erlang EUnit
    \item Python UnitTests
\end{itemize*}

Schalgt nur einer der Tests fehl wird der zuletzt eingespielte Code
zur�ckgesetzt. Es ist zu beachten, dass nicht nur die Masse an Tests, sprich
die Testabdeckung, wichtig f�r IMVU ist, sondern auch die Qualit�t der Tests.
Ein weiteres Merkmal ihrer Firmenkultur ist das Schreiben von qualitativ
hochwertigen Tests. Durch diese Ma�nahmen, also dem Schreiben von gr�ndlichen
hochwertigen Tests, welche sich auf alle Aufgabenbereiche verteilen, schafft
es IMVU ein separates Qualit�tssicherungsteam �berfl�ssig zu machen.

Nachdem s�mtliche Tests erfolgreich durchgef�hrt wurden, wird ein eigen
entwickeltes Build-Skript angesto�en um den neuen Code in die
Produktionsumgebung einzuspielen. IMVUs Produktionsumgebung besteht aus einem
Custer mit derzeit zirka 700 Servern. Das Build-Skript verteilt zwar den Code
im gesamten Cluster, umgestellt werden zun�chst aber nur eine geisse Prozent
Anzahl an Servern. Die Umstellung auf den neuen Code erfolgt recht simpel per
Symlink. 

Durch ein st�ndig aktives Monitoring werden fortlaufend Messwerte �ber den
Gesundheitszustand des Clusters gesammelt. Diese Messdaten beinhalten Werte
f�r CUP-, Speicher- und Netzwerk-Last, aber auch Business Metriken kommen zum
Einsatz. Findet nach einer gewissen Zeitspanne keine Regression des Clusters
statt wird der neue Code auf allen Servern aktiv geschalten. Durch das
begleitende Monitoring k�nnte so noch immer jederzeit auf die vorherige
Version zur�ckgewechselt werden.

Kurz sei noch erw�hnt wie bei IMVU mit den ralationalen Datenbanken verfahren
wird. Da ein Datenbank Schema Rollback nur schwer m�glich ist, bzw. das
Ver�ndern des Schemas einen schwerwiegenden Eingriff darstellt, durchlaufen
Schemamodifikationen, im gegensatz zum Code Deployment, einen formalen Review
Prozess. M�ssen tats�chlich die Strukturen der Tabellen angepasst werden,
bleiben die  alten Tabellen weiterhin bestehen und es werden einfach neue
Tabellen mit der neuen Struktur erstellt. Die Daten werden dann per \emph{Copy
on Read} bzw. per Hintergrund-Job migriert.


\subsection{Huitale}
%http://books.google.co.uk/books?id=G3zTzENmyFIC&lpg=PA111&ots=CuQj-1x2oU&lr&pg=PA111#v=onepage&q&f=false

\subsection{Digg~4}
%http://about.digg.com/blog/continuous-deployment-code-review-and-pre-tested-commits-digg4

\subsection{WiredReach}
%http://www.justin.tv/startuplessonslearned/b/262656299
%http://www.startuplessonslearned.com/2010/01/case-study-continuous-deployment-makes.html

\subsection{Wealthfront}
%http://www.justin.tv/startuplessonslearned/b/286511488?utm_campaign=archive_embed_click&utm_source=www-cdn.justin.tv