\subsection{Jahr auf Quartal}
\label{subsec:jahr-auf-quartal}

In Unternehmen die nur einmal im Jahr eine neue Version ihres Produktes
bereitstellen, kommen häufig lineare, nicht iterative Vorgehensmodelle zum
Einsatz. Eines der bekanntesten Beispiele dafür ist das Wasserfallmodell, dass
aus verschiedenen Phasen wie z.B.: der Analyse, dem Entwurf, der Realisierung
(Implementierung) und dem Testen besteht. Wenn man diese Art von langwierigen
Prozess verfolgt, dann ist es für ein Unternehmen relativ schwer auf
Veränderungen am weltweiten Markt rasch reagieren zu können. Aktualisiert man
nur einmal im Jahr das Produkt läuft man Gefahr, dass die verwendete
Technologie höchstwahrscheinlich schon längst wieder veraltet ist. Daher muss
ein Unternehmen um am Markt konkurrenzfähig zu bleiben, auf kürzere Release-
Zyklen setzen.

Die dabei nächst kürzere Iterationslänge würde z.B.: drei Monate betragen.
Dabei könnte ein Unternehmen einmal im Quartal eine neue Version ihrer
Software bereitstellen. Die große Herausforderung ist die bisherigen Prozesse
so anzupassen, um dieselben Probleme in kürzerer Zeit zu lösen. Eine
Vorgehensweise die dabei offensichtlich nicht funktionieren würde, ist nichts
im Unternehmen zu verändern, jedoch alle bisherigen Praktiken nun in drei
Monaten durchzuführen. Da diese Art der Komprimierung nicht funktionieren
kann, müssen daher fundamentale Techniken adaptiert werden um bestimmte
Aufgaben zu gewissen Zeitpunkten in der kürzeren Iteration durchzuführen.

\subsubsection{Automatisierte Akzeptanztests}
\label{minisec:automatisierte-akzeptanztests}
Manuelles Testen der Software ist zeitaufwendig und fehleranfällig. Hat man
eine Iterationslänge von einem Jahr, kann man jedoch ohne weiteres diese Art
von Testen durchführen. Nach Beendigung der Implementierungsphase wird die
Software an die QA-Abteilung weitergeleitet, die dann mit dem ausführlichen
verifizieren des Produkts beschäftigt ist. Hat ein Unternehmen jedoch nur drei
Monate Zeit, dann wäre es zu aufwändig diesen Prozess jedes Quartal
wiederholen zu müssen. Die Automatisierung einer Regressions-Test-Suite, die
auf \emph{Knopfdruck} ausgeführt werden kann, hilft dem Unternehmen dabei
manuelle Tests aus dem Entwicklungsprozess zu entfernen. Dabei erhalten die
Entwickler nach kürzester Zeit Feedback über das Funktionsverhalten des
Systems. Diese Tests können in Form von Akzeptanztests realisiert werden. Bei
einer Iterationslänge von drei Monaten muss die endgültige Form bzw.
Geschwindigkeit der Tests nicht perfekt oder äußerst schnell sein. Wesentlich
ist nur, dass diese Tests automatisiert durchgeführt werden können. Die
entstehenden Wartezeiten sind dabei für den Projekterfolg nicht kritisch.

\subsubsection{Refactoring}
\label{minisec:refactoring}
Im Wasserfallmodell gibt es eine eigene Phase für den Entwurf der Software.
Möchte man alle drei Monate eine neue Version der Software bereitstellen, muss
das Designen der Software-Architektur über die gesamte Iteration verteilt
werden. Die Zeit für eine eigene Entwurfs-Phase steht bei dieser
Iterationslänge nicht zur Verfügung. Da es jetzt kein \emph{Big Design Up-
Front} geben kann, und das Design kontinuierlich den Gegebenheiten angepasst
werden muss, müssen Softwareentwickler die Technik des \emph{Refactorings}
ausgezeichnet beherrschen. Dadurch wird es möglich, große Design-Änderungen in
kleinen, sicheren Schritten durchzuführen ohne jedoch dabei das
Systemverhalten zu verändern. Entwickler übernehmen dabei die Verantwortung,
regelmäßig in den Softwareentwurf des Produkts zu investieren.

\subsubsection{Continuous Integration}
\label{minisec:continuous-integration}
Auch bei dieser Technik gilt dieselbe Argumentation wie auch schon beim
Refactoring. In einer dreimonatigen Iteration bleibt keine Zeit für eine
eigene Integrationsphase der Software. Sollte es z.B.: kurz vor einem neuen
Release zu Problemen bezüglich des Zusammenbaus des Produkts kommen, die von
den Entwicklern während der Implementierung nicht berücksichtigt wurden,
könnte eventuell ein weiterer Release-Zyklus erforderlich sein, diese um all
diese Probleme zu beheben. Deshalb muss eine Möglichkeit für die
kontinuierliche Integration der Software in Form eines Build-Servers
geschaffen werden, auf dem die Entwickler täglich ihre erledigten Aufgaben
hinzufügen können. Dadurch kann es am Ende der Iteration zu keinen
überraschenden Komplikationen bezüglich des Gesamtprodukts kommen.

\subsubsection{Subscription Modell}
\label{minisec:subscription-modell}
Muss ein Unternehmen nur einmal im Jahr den potentiellen Endkunden von der
neuen Version der Software überzeugen, fällt dieses Vorgehen alle drei Monate
deutlich schwieriger aus. Man kann den Kunden als Unternehmen nicht dazu
bringen, jedes Quartal für eine Aktualisierung der Software erneut zahlen zu
lassen. Sollte das Geschäftsmodell der Firma jedoch vorsehen, dass der Kunde
für Upgrades des Produkts zahlen muss, dann kann die Iterationslänge nicht auf
häufigere Releases umgestellt werden. Leider verliert man dadurch auch die
Vorteile des häufigeren Feedbacks des Benutzers, genauer gesagt all die
Informationen die man aufgrund der Benutzung des Produkts durch den Endkunden
und die Entwicklung des Produkts am Markt erhält. Daher muss das
Geschäftsmodell des Unternehmens ebenfalls angepasst werden und möglicherweise
eine Form von Subscription-Modell eingeführt werden. Dabei zahlt der Kunde
einmal im Jahr einen Pauschalbetrag und erhält sämtliche Upgrades der Software
ohne weitere Bezahlung. Diese Art von Geschäftsmodell ist absolut kritisch für
den Erfolg für eine Umstellung auf dreimonatige Release-Zyklen.
