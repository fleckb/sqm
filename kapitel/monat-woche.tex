\subsection{Monat auf Woche}
\label{subsec:monat-auf-woche}

Es kann natürlich vorkommen, dass Kunden selbst für einen monatlichen Release-
Zyklus zu ungeduldig sind und deshalb auf wöchentliche Updates ihrer Software
bestehen. D.h. das Entwickler-Team steht wieder vor der Aufgabe, dieselben
Probleme in noch kürzer Zeit zu bewältigen. Die erste Frage die sich bei der
Umstellung auf wöchentliche Release-Zyklen stellt, wann soll die Software
bereitgestellt werden? Freitag?

Nein, der letzte Tag der Woche ist ein äußerst schlechter Tag um neue
Produktaktualisierungen dem Kunden zur Verfügung zu stellen, genauso am Tag
bevor Weihnachten. Die Mitte der Woche scheint eine gute Idee zu sein. D.h.
Jeden Mittwoch wird eine neue Version der Software bereitgestellt, sodass der
Kunde mit dieser seine Arbeit verrichten kann. Dafür müssen wieder einmal die
Organisation und der Zeitpunkt der Lösung angepasst werden um einen
einwöchigen Release-Zyklus gerecht zu werden. Das Entwickler-Team also fünf
Tage Zeit um ein für den Kunden wertvolles Inkrement zu erstellen. Wie soll
man das anstellen? Erster Schritt ist eine Atomated Data Migration.
Automatische Datenmigration

Bei einem einmonatigen Release-Zyklus hat man ein paar Tage Zeit um das DB-
Schema anzupassen. Keinem Entwickler mach diese Art von Spaß, jedoch bleibt
genügend Zeit um es manell anzupassen. Innerhalb von einer Woche geht sich
diese Arbeit allerdings nicht aus. Manuelle Anpassungen um Datenmigrationen
durchzuführen sind einfach zu kostspielig. Eine voll automatisierte \emph{1
-Button-Push} Datenmigration ist unabdingbar um Daten von einem alten zu einem
neuen Zustand überzuführen.

Falls man einen Service 24/7 bereitstellen muss, dann muss man noch
zusätzliche Dinge tun. Mehr davon bei täglichen Releases. Bei dieser Phase
darf Datenmigration auf jeden Fall kein Problem mehr darstellen. Vor allem
handelt es sich hierbei nur um ein Engineering-Problem. Dabei unterstützt ein
Datastore automatisierte Datenmigrierung vielleicht besser als ein anderer.
Dafür ist ja Engineering da, man nimmt die Zutaten die man hat und wendet sie
passend am bestehenden Problem an.

\subsubsection{Temporäre Branches}
In einem einmonatigen Zyklus können Entwickler Branches für Änderungen der
Software anlegen und diese nicht unmittelbar wieder in den Main-Branch zurück
mergen. D.h. die Programmierer arbeiten z.B. über einen Zeitraum von einer
Woche an einer Änderung in einem neu erstellten Branch und migrieren diesen
dann nach Fertigstellung wieder zurück in den Trunk.

Dabei kann es zu komplizierteren Merges kommen, die einiges an Zeit
beanspruchen. Genau diese Zeit ist in einem einwöchigen Zyklus nicht mehr
vorhanden. Daher können Entwickler maximal temporäre Branches anlegen, die
maximal nach ein paar Stunden wieder in den Main-Branch zurückmigriert werden.
Dadurch verhindert man komplizierte Merges die aufgrund einer hohen Anzahl von
Konflikten auftreten könnten.

Es stellt sich die Frage, ob bei temporären Branches nur halbfertige
Änderungen wieder in die Main-Line zurückmigriert werden. Jedoch überwiegt die
Verhinderung der auftretenden Kosten bei komplizierten Merges diese Tatsache.
D.h. man hat keine andere Wahl als kurzweilige, temporäre Branches anzulegen
da ansonsten einwöchige Release-Zyklen keine Chance auf Erfolg hätten.

Ein wichtige Begabung die Entwickler bei immer kürzeren Zyklenen haben sollte,
ist die immer feinere Unterteilung von Arbeiten, sodass die Software nach
einer Änderung trotzdem noch funktioniert. Z.B. hat man ein Refactoring von
ca. 4000 LoC vor sich. Diese Änderungen müssen nicht alle auf einmal
geschehen. Die Begabung eines Entwickler liegt nun bei der Unterteilung der
vorzunehmenden Änderungen. Diese müssen so proportioniert werden, sodass nach
einer Integrierung die Software noch voll funktionsfähig ist. Das Problem ist
also die Reihenfolge der Änderungen sorgfältig zu bestimmen.

\subsubsection{Keystoning}
Hat man das Problem, dass ein zu implementierendes Feature mehr als eine Woche
Zeit in Anspruch nimmt, man hat allerdings wöchentliche Releases, kann man
zuerst die notwendige Funktionalität erstellen, die der Endbenutzer nicht zu
Gesicht bekommt. Im nächsten Release-Zyklus kann man dann die notwendigen UI-
Anpassungen durchführen und

gewährleistet dadurch die Einhaltung der fünftägigen Software-Aktualisierung.
D.h. man sollte bei einwöchigen Releases die sichtbaren Änderungen immer zum
Schluss durchführen. Natürlich entsteht dadurch ein gewisses Risiko, da man ja
Code in Produktion ausgeliefert hat, der nicht aufgerufen werden kann. Diesen
Preis muss man aber für wöchentliche Releases bezahlen.

\subsubsection{Kanban}
Die Vorteile von Kanban werden bei wöchentlichen Release-Zyklen so richtig
wertvoll. Genau wie all die bereits erwähnten Techniken kann Kanban natürlich
auch bei länger andauernden Phasen zum Einsatz kommen. Bei einwöchigen
Iterationen macht eine Art des Pull-Modells auf jeden Fall Sinn. Hat man in
jedem Release-Zyklus eine fixe Anzahl von Tasks die zu erledigen sind und
dieser Plan kann aufgrund von irgendwelchen Einflüssen nicht eingehalten
werden, dann kann das zu Folgefehlern führen, sodass der Gesamtplan gefährdet
ist.

Im Gegensatz dazu bietet Kanban die Möglichkeit mittels des Pull-Modells, dass
ein Entwickler z.B. am Montag in der Früh eine neue Aufgabe vom Aufgaben-Stack
entnimmt und diese abarbeitet. Sollte dieser Task mehr Zeit als angenommen
beanspruchen, gefährdet dies keine anderen Aufgaben. Dadurch müssen keine
Planänderungen durchgeführt bzw. keine entsprechenden Meetings einberufen
werden.

\subsubsection{One-Button-Deploy}

Will man wöchentliche Release-Zyklen realisieren wird ein \emph{One-Button-
Deploy} unvermeidlich. Selbst wenn man einen vollen Monat Zeit hat, wird ein
zweistündiger, manueller Deploy, bei dem noch dazu hin und wieder Fehler
auftreten können, wirklich zur Qual für jeden Entwickler. Wenn man sich jetzt
vorstellt, dass man für jeden Release diese fehleranfällige Arbeit erledigen
muss, wird man schnell einsichtig, dass die Zeit dafür zu kostbar ist.

Auch hier gilt wieder, dass ein \emph{One-Button-Release} ebenfalls für
jährliche Entwicklungs-Phasen eingesetzt werden kann. Jedoch ist diese Technik
bei dieser Geschwindigkeit nicht kritisch. Bei wöchentlichen Releases ist es
auf jeden Fall kritisch.

\subsubsection{Kein separates Test Team}
In wöchentlichen Release-Zyklen unterscheidet man nicht mehr zwischen einem
separaten Test-Team und einem Entwickler-Team. Selbst dieser geringe Abstand
zwischen den beiden Teams wir in dieser kurzen Zeit zu aufwendig um ihn zu
unterstützen. Der typische Arbeitsablauf bei dem ein Entwickler dem Tester neu
hinzugefügte Funktionalität zum Testen bereitstellt, und dieser dann
selbstständig das erwartete Verhalten der Software verifiziert.

Der Entwickler erhält dann unmittelbares Feedback vom Tester auf das er dann
in entsprechender Art und Weise reagieren muss. Dieses Request-Response
Verfahren ist zu zeitintensiv um in einer Woche eingesetzt zu werden. D.h. es
gibt das Team das sowohl neue Funktionalität bereitstellt, also auch Feedback
für diese erzeugt, auf das das Team entsprechend reagieren muss. Somit gilt,
das Team muss für Implementierung und Testen gemeinsam Verantwortung
übernehmen.

\subsubsection{Keine Up-front Usability}
Jeder Spezialbereich der Software-Entwicklung hat das Verlangen vor allen
anderen berücksichtigt zu werden. Z.B. wird gerne die Software-Architektur als
wichtigstes Kriterium für die erfolgreiche Erstellung des Produkts angesehen.
Die Architektur sollte zu Beginn entworfen werden, danach kann die restliche,
oft als nicht so wesentlich empfundene Arbeit umgesetzt werden. Genau
dieselben Ansichten haben die Designer, die Tester, etc. In wöchentlichen
Release-Zyklen kann diese Kurzsichtigkeit nicht unterstützt werden.

D.h. Entwickler können nicht fünf Tage auf das Architekturdokument warten,
wenn man nur genau eine Woche Zeit hat, die gewünschten Features
bereitzustellen. Daher muss die Aufgabe des Usability-Designs über die
komplette Phase verteilt umgesetzt werden im Gegensatz zu einer Aufgabe, die
nur zu Beginn des Zyklus stattfindet.

\subsubsection{Aktiver Release-Branch}
Bei wöchentlichen Iterationen wird die Pflege eines separaten Release-Banches,
der aktiv verändert wird, zu zeitaufwändig und muss daher entfernt werden. Hat
man einen Monat Zeit, und man muss eventuell einen Software-Patch einspielen
der z.B. am 1. eines Monats deployt worden ist, ist es nicht kritischen diesen
Patch auch im Entwickler-Branch nachzuziehen.

D.h. in diesem Fall sind zwei Branches noch ohne Weiteres umsetzbar. Bei
fünftätigen Zyklen wird es zu aufwendig den Release-Brach aktiv zu
aktualisieren, da man ja ständig die Software verändert. Jede Änderung doppelt
zu warten ist in wöchentlichen Iterationen zu kostspielig. D.h. Kunden müssen
sich eine Woche, dafür maximal eine Woche, auf einen Bug-Fix gedulden.
