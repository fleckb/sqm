\section{Einleitung}
\label{sec:einleitung}
Softwareunternehmen haben das Ziel Produkte zu schaffen, die ihre potentiellen Abnehmer bei ihrer Tätigkeit, sei es nun geschäftlich oder privat hinsichtlich einer Produktivitätssteigerung zu unterstützen. Dazu muss der Hersteller die genauen Arbeitsabläufe des Käufers genau ermitteln, analysieren und dann versuchen zu optimieren. Das funktioniert nur in engster Zusammenarbeit mit dem Kunden. Das Problem was bei der Softwareentwicklung allerdings heutzutage nach wie vor besteht, ist, dass der potentielle Abnehmer selbst oft nicht weiß, wie das Produkt aussehen soll beziehungsweise was es genau können muss. 

Deshalb ist es für den Hersteller ungemein wichtig, dem Kunden so früh wie möglich eine funktionstüchtige erste Version der Software bereitzustellen. Dass diese natürlich nicht bereits die komplette Anzahl aller Features implementiert haben muss, ist an dieser Stelle klar. Jedoch ist es für den Kunden unheimlich hilfreich, wenn er zum Beispiel an einem Prototypen des Produkts gewisse Arbeitsabläufe ausprobieren kann. Dabei wird er sehr schnell feststellen ob ihm diese Version der Benutzeroberfläche gefällt oder nicht. 

Das heißt, je früher der Käufer etwas Testbares zum Ausprobieren bekommt, desto besser für ihn als auch für den Softwarehersteller. Einerseits verfeinert der Kunde dadurch die Vorstellung darüber was er eigentlich möchte, und andererseits können die Softwareentwickler ihren Entwicklungsprozess darauf anpassen, wie sie am besten mit frühen Änderungswünschen umgehen können. Stellt sich nun die Frage, was genau die Probleme bei dem zuvor beschrieben Ablauf sein könnten. 

Einerseits müssen die Erwartungen des Kunden bestmöglich durch das Softwareunternehmen zufriedengestellt werden. Eine Voraussetzung dafür ist so früh wie möglich im Produktlebenszyklus Feedback des Abnehmers einzuholen, um für diesen die optimalste, nämlich auf seine Bedürfnisse angepasste, Software zu produzieren. Das heißt die Geschwindigkeit in der das erste (und auch alle weiteren Releases des Produkts) bereitgestellt werden, ist von enormer Bedeutung. 

Wie sieht es nun in der Praxis tatsächlich aus? In was für Abständen werden Softwareprodukte zum Beispiel von großen Softwareherstellen wie Microsoft, Apple etc. am Markt veröffentlicht. Kann diese Art von Software überhaupt mit Enterprise-Software verglichen werden? Wenn man aktuelle Bücher über Software-Engineering und Vorgehensmodelle in der Softwareentwicklung studiert, werden keine genauen Beschreibungen und Vorgaben zu Iterationslängen gemacht. Hin und wieder liest man in aktuellerer Lektüre, dass Releasezyklen von zwei bis vier Wochen empfehlenswert wären, um sinnvolle, nämlich für den Kunden wertvolle, Funktionalität von Software bereitzustellen. 

Bedeutet das nun, dass es in der Art und Weise wie Software heutzutage entwickelt wird, bei Iterationslängen nicht unterschieden werden muss? Kann ein Unternehmen, das einmal jährlich eine neue Version ihrer Software veröffentlicht rein theoretisch die Releasezyklen auf einen Monat verkürzen? Wenn man zwei ausgebildete Informatiker mit Hochschulabschluss darüber befragt, was denn genau die beste Methodik sei, um effektiv und effizient Software zu entwickeln, kann es sein, dass man zwei komplett verschiedene Antworten bekommt. Handelt es sich vielleicht dabei um die rapide fortschreitende Technologie die so viele unterschiedliche Vorgehensweisen zur Entwicklung von Software unterstützt? Warum sollte ein Unternehmen nur einmal im Jahr eine neue Version ihres Produkts am Markt bringen, wenn es möglich wäre, die Kunden jede Woche oder sogar jeden Tag mit einer Verbesserung  der Software zu beglücken? 

Tatsächlich handelt es sich bei der Softwareentwicklung um einen komplexen Prozess der bis heute noch immer auf unterschiedlichste Weise durchgeführt wird. Es wurde noch kein Vorgehensmodell erfunden oder keine Technologie erschaffen, die einen einheitlichen, formalen Entwicklungsprozess von Software ermöglicht. Entwickler müssen nicht nur die Kunst des Software-Engineerings beherrschen, was für sich schon jahrelange Erfahrung beansprucht, sondern auch im Team mit anderen Menschen zusammenarbeiten können, was ebenfalls eine große Herausforderung in vielen gängigen Vorgehensmodellen darstellt. Außerdem muss die Problemdomäne vom Entwickler genau verstanden werden, denn im Endeffekt ist das Produkt ein Hilfsmittel das den Arbeitsprozess des Kunden erleichtern soll. 

Das kann natürlich nur umgesetzt werden, wenn der Softwarehersteller in enger Zusammenarbeit mit dem Softwareabnehmer in einer einheitlichen Sprache, nämlich die des Kunden, den Problembereich genau analysieren kann. Wenn man all diese Punkte im Vorgehensmodell des Softwareentwicklungsprozesses berücksichtigen möchte (es gibt natürlich noch viele weitere), fängt man an zu verstehen, warum heutzutage noch in so unterschiedlich langen Releasezyklen Software entwickelt wird. Prinzipiell gilt es für die Softwareentwickler die gleichen Probleme in unterschiedlichen Zeitspannen zu lösen. Dabei muss ein Produkt entstehen das den Ansprüchen des Kunden gerecht wird. 

Generell gilt, hohe Qualität für wenig Geld in möglichst kurzer Zeit in Software umzuwandeln. Dabei versteht man unter Qualität die Umsetzung von funktionalen als auch nichtfunktionale Anforderungen des Käufers. Dazu gehören zum Beispiel, dass das ausgelieferte Produkt so wenig Fehler wie möglich enthält, dass es performant ist, dass es ein ansprechendes Äußeres besitzt und natürlich muss es die gewünschten Features beinhalten. Dass natürlich nicht alle Aspekte (Qualität, Geld, Zeit) gleichermaßen gut optimiert werden können, scheint an dieser Stelle nun klar, sonst gäbe es ja nur Softwareunternehmen die stündlich neue Versionen ihrer Produkte kostenlos in höchster Qualität ausliefern würden. Softwareentwickler müssen also Kompromisse eingehen. Diese müssen jedoch so optimiert werden, dass der Kunde einerseits glücklich mit dem Produkt wird, der Hersteller andererseits Geld dabei verdienen kann. 

Die folgende Arbeit befasst sich mit den Entwicklungstechniken, den Vorgehensweisen und eventuellen Geschäftsmodellen die für verschiedene Iterationslängen notwendig sind, um erfolgreich Produkte unter den gegebenen Voraussetzungen (Qualität, Geld, Zeit) zu realisieren. Dabei werden die Zeitspannen der einzelnen Phasen von einem Jahr, zu einem Realease innerhalb eines Quartals, zu einem Monat und zu einem Tag verkürzt. Dabei gilt es stets die Feedbackzyklen innerhalb dieser Iteration zu optimieren. Zum Beispiel kann das durch Automatisierungsprozesse in jeglicher Hinsicht geschehen. Es gilt dabei manuelle Arbeiten so weit wie möglich aus dem Entwicklungsprozess zu entfernen, die unmittelbare Kommunikation innerhalb des Teams zu erhöhen und eine wirkliche Vertrauensbasis zwischen Endkunden und Softwareentwickler zu schaffen. Hohe Produktqualität im Sinne von Fehlerfreiheit wir dabei durch umfangreiche Testvorgänge sichergestellt. Feedback und Kommunikation sind unter anderem die mitunter wichtigsten Eigenschaften eines effektiven und effizienten Vorgehensmodells.